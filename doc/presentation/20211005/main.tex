% !TEX root=./main.tex

\documentclass[aspectratio=169]{beamer}

\usecolortheme{seahorse}

\usepackage{standalone}
\usepackage{amsmath} % for matrix example
\usepackage{upquote} % so single quotes can be directly pasted
\usepackage{xcolor}
\usepackage{graphicx}
\usepackage{multimedia}
\usepackage{pgfpages} % for seperating pages
\usepackage{algorithm,algorithmic}
\usepackage{minted}
\usepackage[thinc]{esdiff}

\setbeameroption{show notes}
\setbeameroption{show notes on second screen=right}

\title{"Large Steps in Inverse Rendering of Geometry"}
\subtitle{Baptiste Nicolet, Alec Jacobson, and Wenzel Jakob, \\ Proceedings of SIGGRAPH Asia 2021}
\author{Hyeonjang An}

\institute{
    Photorealistic Rendering \\
    GIST
}

\begin{document}
\setbeamertemplate{caption}{\raggedright\insertcaption\par}
\begin{frame}
\titlepage

\note[item]{
    I am going to introduce the new paper from Siggraph Asia 2021.
    The paper title is "Large steps in inverse rendering of geometry".

    This guy, Wenzel Jakob, as we already know,  is the big guy in Rendering in Lausanne Technical University

    And this guy, alec Jacobson is also big guy in Geometry Processing. Assistant professor in Toronto university.

    Anyway, let me start.
}

\end{frame}

\begin{frame}{Outline}
    \tableofcontents


    \note[item] {
        This is the outline of my presentation
    }

\end{frame}

\section{Problem statement}
\section{Background}   
    \subsection{Differentiable Rendering}
    \subsection{laplace operator}
\section{Method}
    \subsection{\alert{Modifed Gradient Descent}}
    \subsection{Reparameterization}
    \subsection{Momentum and Variance}
    \subsection{Remeshing}
\section{Result}
\section*{Conclusion}

% 1. Problem statement
% !TEX root=../main.tex
\documentclass[beamer]{standalone}
\begin{document}

% introduction
\begin{frame}{Problem statement and Idea}
    \begin{itemize}
        \setlength\itemsep{1em}

        \item Problem 
        \begin{enumerate}
            \item ill-posed problem
            \item noisy gradient descent
            \begin{itemize}
                \item Monte-carlo differentiable rendering
                \item Stochastic gradient descent
            \end{itemize}
        \end{enumerate}
        \item Motivation
        \begin{enumerate}
            \item Observing gradient filtering behaviour on the joint optimization problem \\ 
            (texture and envmap)
        \end{enumerate}
        \begin{figure}
            \includegraphics{./figures/Intro-1.png}
        \end{figure}
    \end{itemize}
\end{frame}
\end{document}

% 2. Background explanation
% !TEX root=../main.tex
\documentclass[beamer]{standalone}
\begin{document}

% Differentiable rendering
\begin{frame}{Background}
\framesubtitle{Differentiable rendering}

\begin{itemize}
    \item Differentiable rendering pipeline
    \begin{figure}[h]
        \includegraphics[width=0.8\linewidth]{figures/background-figure-2.jpeg}
        \caption{Mitsuba2: inverse rendering}
    \end{figure}
    \item Objective function
    \begin{enumerate}
        \item $ argmin \frac{1}{2} \left\lVert G.T. - R(x) \right\rVert^{2}  $
    \end{enumerate}

\end{itemize}

\note[item] {
    So, to understand this paper, we need at least two prior knowledges. The first one is differentiable rendering. 
    I will pass this part, because the previous presenter explain this sufficiently.
    }
\end{frame}

% Mesh propertices / laplace
\begin{frame}{Background}
\framesubtitle{Laplace operator}
\begin{itemize}
    \item Laplace operator
    \begin{enumerate}
        \item $ \mbox{\small $\Delta f = \diffp{f}{{x_{1}^{2}}} + \diffp{f}{{x_{2}^{2}}} + \cdots + \diffp{f}{{x_{n}^{2}}}$ } $
        \item Laplacian is deviation from local average
    \end{enumerate}

    \pause

    \item Discrete laplace operator on \small polygonal mesh $\mathcal{M} = (V, E)$
    \begin{enumerate}
        \item The weights $w_{ij} \in \mathbb{R} $ discretize the first derivative along an edge
        \item The addition of first derivatives within $L$ extends a second derivative to signals on $\mathcal{M}$
    \end{enumerate}
    \begin{figure}[h]
        \includegraphics[width=0.8\linewidth]{figures/background-figure-3.png}
    \end{figure}

    \pause

    \item Laplace regularaizer
    \begin{equation}
        minimize_{x\in\mathbb{R}^{nX3}}  \; \Phi(R(x)) + \textcolor{red}{\frac{\lambda}{2} tr(x^{T}Lx)}
    \end{equation}
\end{itemize}
    
% notes %
\note[item]{
    The second one is laplace operator.

    The laplace operator looks like this. To easily understand, laplacian indicate the deviation from local average.
    That means, in the specific point, how spiky that point is along the local area.

    On polygonal mesh, we can think laplacian similar as Graph Laplacian.
    On Laplace operator, the weights is the discretized version of the first derivatie along the an edge.
    Therefore, we can interpret the addition of first derivative of laplacian as a second derivate of signal on mesh.
    
    Then, it can be said that a laplace regularizer wants each vertex to be at the center of its neighbors.
    So it is a compromise for the original objective function.
    }   
\end{frame}

\end{document}


% 3-1.method part
% !TEX root=../main.tex
\documentclass[beamer]{standalone}
\begin{document}
% ==============================================================
% Method intro
% ==============================================================
\begin{frame}{Method}
    \begin{itemize}
    \item Main goal is to determine a methodology 
    for taking large descent steps in mesh optimizing problems 
    using differentiable rendering
    \begin{figure}
        \begin{equation}
            minimize_{x\in\mathbb{R}^{nX3}}  \; \Phi(R(x))
        \end{equation}
    \caption{where $x$ is collection of vertex positions, $R$ is rendering function, $\Phi$ is a loss function}
    \end{figure}
\end{itemize}

% note %
\note[item]{
    As before said, the main goal of this paper is showing how to determine a methodology for taking large steps.
    So, the objective function could be like here.
    x is the vertex positions, R is rendering function and phi is loss function
}
\end{frame}



\end{document}

% 3-2 method part
% !TEX root=../main.tex
\documentclass[beamer]{standalone}
\begin{document}
% ==============================================================
% Method : Gradient Descenet Explanations - 0 
% ==============================================================   
\begin{frame}{Method}
    \framesubtitle{Laplacian operator on the image kernel}
        \begin{figure}[t]
        \centering
            \includegraphics[width=6cm]{./figures/image-kernel-laplacian.png}
            \caption{from: \emph{"A Tour of Modern Image Filtering"}}
        \end{figure}
        \begin{itemize}
            \item Filtering problem, the convenient vector form $z(x)=Wy$
        \end{itemize}
        \begin{table}
            \centering
            \resizebox{12cm}{!}{
            \begin{tabular}{ | c | c | c | c | } \hline
                Graph Laplacian & Symmetric                & DC eigenvector & Spectral Range \\ \hline
                Un-normalized   & $D - K$                  & Yes            & [0, n] \\ \hline
                Normalized      & $I - D^{-1/2}KD^{-1/2}$  & No             & [0, 2] \\ \hline  
            \end{tabular}}
        \end{table}

    % note %
    \note[item] {
    }
\end{frame}

\begin{frame}{Method}
    \framesubtitle{The choice on step length}
        \begin{equation}
            x \leftarrow x - \eta \textcolor{red}{(I + \lambda L)^{-p}} {\diffp{\Phi}{x}}
        \end{equation}
        \begin{itemize}    
            \item Need for choosing step size $\lambda$
            \item Step size can be chosen with weak wolfe condiiton, goldstein condition and backtracking ..
            \begin{itemize}
                \item But, we need the function evaluation for each step length
                \item In the current case, our function is MSE between target and rendering image
            \end{itemize}
            \item Now just using the scaled step size from an original adam optimizer
            \begin{itemize}
                \item On the "Large steps in the inverse rendering of Geometry" $\lambda$ is set as from 15 to 50
                \item On the "Laplacian Smooth Gradient Descent", the authors shows a proof on optimal value
            \end{itemize}
        \end{itemize}
\end{frame}

\end{document}

% 3-3 method part
% !TEX root=../main.tex
\documentclass[beamer]{standalone}
\begin{document}
% ==============================================================
% Method : Gradient Descenet Explanations - 7 (Original Adam vs Uniform Adam)
% ============================================================== 
\begin{frame}{Method}
\framesubtitle{Momentum and Variance}

\begin{columns}[T] % align columns
\begin{column}{.45\textwidth}
\begin{algorithm}[H]
    \begin{algorithmic}[1]
        \WHILE{$x$ not converged do}
            \STATE $g \leftarrow (I + \lambda L)^{-1} \diffp{\Phi}{x}$
            \STATE $m_{1} \leftarrow \beta_{1} m_{1} + (1-\beta_{1})g$
            \STATE $m_{2} \leftarrow \beta_{1} m_{2} + (1-\beta_{1})g^2$
            \STATE $u \leftarrow u-\frac{\eta}{(1-\beta_{1}^{k}) \sqrt{\frac{\left\lVert m_{2} \right\rVert_{\infty}}{1-\beta_{2}^{k}}}}$
        \ENDWHILE
        \STATE where $x(u)=(I+\lambda L)^{-1}u$
        \end{algorithmic}
    \caption{UniformAdam}
    \label{alg:seq}
    \end{algorithm}
\end{column}
\begin{column}{0.45\textwidth}
    \begin{figure}
        \includegraphics[width=\textwidth]{figures/method3-figure-1.png}
    \end{figure}
\end{column}
\end{columns}

% note %
\note[item] {
    Until now, I have explained the main idea of this paper.

    This is final form their implementation of Adam.

    The authors said that they discovered that aggresive steps disturb the smoothness of result.
    So, they modify it to a more uniform adaptation, and the picture of rightside is the comparison.
}
\end{frame}

% ==============================================================
% Method : Gradient Descenet Explanations - other details (implementation, )
% ============================================================== 
\begin{frame}{Method}
\framesubtitle{Remeshing}
\begin{itemize}
    \item Remeshing
    \begin{enumerate}
        \item remeshing is the technique that rebalance the energy of mesh
        \item In this paper, the authors adapted isotropic remeshing (Bostch and Kobbelt [2004])
        \item remeshing at manually specified timesteps
    \end{enumerate}
    \begin{figure}
        \includegraphics[width=0.5\textwidth]{figures/method3-figure-2.png}
    \end{figure}
\end{itemize}

% note %
\note[item] {
    And lastly, they use remeshing technique to solve the mesh energy balancing and resolution problem.
    
    Remeshing is the technique that rebalance the energy of mesh, by resampling vertices and modifying edge length.
    There are several remeshing technique, in this paper they adapt isotropic remeshing from Bostch and Kobbelt, the original paper title is "A Remeshing Approach to Multiresolution Modeling"
} 

\end{frame}
\end{document}

% 4. results part
% !TEX root=../main.tex
\documentclass[beamer]{standalone}
\begin{document}
% ===========================================
% Results - Video
% ===========================================
\begin{frame}{Results}
    \framesubtitle{Video}

\note[item] {
    Here is the video result.
}

\end{frame}

% ===========================================
% Results - Analysis
% ===========================================
% \begin{frame}{Results}
% \framesubtitle{Shape reconstruction comparison}
%     \begin{figure}
%         \includegraphics[height=0.85\textheight]{figures/result-3.png}
%     \end{figure}

% \note[item] {
%     This is the comparison result of inverse shape reconstruction.
% }

% \end{frame}

% ===========================================
% Results - Texture
% ===========================================
\begin{frame}{Results}
\framesubtitle{Texture reconstruction using Monte Carlo sampling}
    \begin{figure}
        \includegraphics[width=0.8\textwidth]{figures/result-4.png}
    \end{figure}

    % note %
\note[item]{
    % @@TODO
    Also, the authors suggest another adaption of this second-order derivative method.

    The suggested example is texture reconstruction in Monte Carlo rendering.

    Because of the noisy gradient, we should take small steps on this optimization sequence or need some regularaization in differentiable path tracing.
    Their method can solve this problem as the point of view on filtering the gradient.
}
\end{frame}
\end{document}

% 5. conclusion
% !TEX root=../main.tex
\documentclass[beamer]{standalone}
\begin{document}
\begin{frame}{Summary}

\begin{itemize}
    \setlength\itemsep{1em}
    \item Gradient filtering using the known propertices on target parameters
    \item Test on the joint optimization problem, texture case
    \item Due to the bias (with the rougly designed gradient filtering), \\ 
    optimizing enviroment map shows a awful result
\end{itemize}
\end{frame}

\begin{frame}{Conclusion}

    \begin{itemize}
        \setlength\itemsep{1em}
        \item Gradient filtering makes the convergence faster
        \begin{itemize}
            \item reduce noise
            \item give a bigger step length
        \end{itemize}
        \item to global minimum, when properly applied 
        \item Not likely mesh, bias becomes bigger problem in the texture optimization procedure
    \end{itemize}
    
        % notes %
        \note[item]{
        }
        
    \end{frame}

    \begin{frame}{Q\&A}

        \note[item] {
            this is the end of the presentation
        }
    \end{frame}
\end{document}

\end{document}