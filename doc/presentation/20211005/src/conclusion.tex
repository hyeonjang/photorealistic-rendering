% !TEX root=../main.tex
\documentclass[beamer]{standalone}
\begin{document}
\begin{frame}{Summary}

\begin{itemize}
    \setlength\itemsep{1em}
    \item Large steps (fast convergence) in Mesh optimizing using differentiable rendering by second-order derivate method
    \item Not add regularization, filter the gradients of the original objective with the laplace.
    \item Any adaptation can be possible, if second-order derivative is possible
\end{itemize}

    % notes %
    \note[item]{

        To summarize, in differentiable rendering, It is hard to optimize vertex position roughly. 
        The authors solve this problem by applying second-order optimization method.
        And this method can be applied in any other problems, if second-order derivative is possible

    }

\end{frame}

\begin{frame}{Conclusion}

    \begin{itemize}
        \setlength\itemsep{1em}
        \item Until now, only first derivative and regularaization are used
        \item It is much better to take second-order derivative when taking mesh propertices
        \item Second-order derivative can be combined with second-order optimization method like Newton's method, which has the faster convergence speed
    \end{itemize}
    
        % notes %
        \note[item]{
            Finally, I have choosed some important points to remember.

            I think these are the most important part of this paper.
            
            Until now, first derivative and regularaization are used.
            But for polygonal mesh, or on mesh deformation problem, we should take second-order derivative propertices to get more high quailty results.
            This second-order derivative can be combined with second-order optimization method like Newton's method. And this method has the faster convergence speed than others.
        }
        
    \end{frame}

    \begin{frame}{Q\&A}

        \note[item] {
            this is the end of the presentation
        }
    \end{frame}
\end{document}