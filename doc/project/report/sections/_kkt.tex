\section*{KKT Conditions}\label{ch:ch4label}
In order to obtain the KKT conditions, we set the gradient of the Lagrangian to zero. $\bigtriangledown L = 0$. We do not have any inequality constraints, so we will not look at complementary slackness. Instead of using the dual variable $y_t$ for the constraint $c_t=l_t$, we will simplify the problem by replacing $l_t$ with $c_t$.
\begin{align}{\label{eq1}}
	\frac{dL}{dc_t} = 0 \implies \beta^t\frac{1}{c_t}-\beta^t(\mu_t+\lambda_t)P_t=0 \implies \frac{1}{c_t}=(\mu_t+\lambda_t)P_t
\end{align}
\begin{align}{\label{eq2}}
	\frac{dL}{dl_t} = 0 \implies -\beta^t\frac{1}{1-l_t}+\beta^t\lambda_t P_t=0 \implies \lambda_t = \frac{1}{(1-l_t)P_t}
\end{align}
\begin{align}{\label{eq3}}
	\frac{dL}{ds_{t+1}} = 0 \implies -\beta^t\lambda_t+\beta^{t+1}\lambda_{t+1}(1+R_{t+1})=0 \implies \lambda_t = \beta \lambda_{t+1}(1+R_{t+1})
\end{align}
\begin{align}{\label{eq4}}
	\frac{dL}{dm_{t+1}} = 0 \implies -\beta^t\lambda_t+\beta^{t+1}(\mu_{t+1}+\lambda_{t+1})=0 \implies \beta(\mu_{t+1}+\lambda_{t+1})=\lambda_t
\end{align}
Using equation (\ref{eq1}) and equation (\ref{eq4}), we get an expression for $\lambda$ and we plug this in equation (\ref{eq2}). We also use the constraint $c_t=l_t$.
\begin{align}{\label{eq5}}
	\lambda_t=\frac{\beta}{c_{t+1}P_{t+1}} = \frac{1}{(1-c_t)P_t} \implies \frac{P_{t+1}}{P_t} = \beta\frac{1-c_t}{c_t+1}
\end{align}
We also use equation (\ref{eq2}) in equation (\ref{eq3}).
\begin{align}{\label{eq6}}
	\frac{1}{(1-c_t)P_t}=\frac{\beta(1+R_{t+1})}{(1-c_{t+1})P_{t+1}} \implies \frac{P_{t+1}}{P_t}=\frac{1-c_t}{1-c_{t+1}}\beta(1+R_{t+1})
\end{align}
Finally, we have $P_t c_t = m_t$.
\begin{align}{\label{eq7}}
	\frac{P_{t+1}}{P_t}=\frac{(1+\tau_t/m_t)c_{t+1}}{c_t}=\frac{1-c_t}{1-c_{t+1}}\beta(1+R_{t+1})=\beta\frac{1-c_t}{c_{t+1}}
\end{align}
Using the equalities in (\ref{eq7}), we find that
\begin{align}{\label{eq8}}
	c_{t+1} = \frac{1}{2+R_{t+1}}
\end{align}
and all three equalities in (\ref{eq7}) are satisfied if $c_{t+1}=c_t \implies c_t =c^*$ is a constant; consequently, $R_t=R$ and $l_t = l^*$ are constants as well, and 
\begin{align}{\label{eq9}}
	1+\frac{\tau_t}{m_t} = \beta(1+R_t) \implies \frac{\tau_t}{m_t}=g \rightarrow \textrm{constant}
\end{align}
For optimal conditions, we set $l_t = c_t = c^*$ in $\sum_{t=0}^{N-1}{\beta^t}[ln(c_t)+ln(1-l_t)]$ and set its derivative to zero, since the objective is reduced to one variable.
\begin{align}{\label{eq10}}
	\frac{d}{dc^*}\frac{1-\beta^N}{1-\beta}[ln(c^*)+ln(1-c^*)]=0 \implies \frac{1}{c^*}-\frac{1}{1-c^*}=0 \implies c^* = 1/2
\end{align}
Since $c^*=\frac{1}{2+R} \implies R = 0$ and $g=\beta-1$. The optimal values are $m_{t+1}^*=\beta m_t \implies m_{t+1} = \beta^(t+1) m_0$, $s_{t+1}=0$, $c_{t} = l_{t} = 1/2$ for the system in equilibrium for N days.