\section*{Related work}\label{ch:ch2label}

\subsection*{Physically based differentiable rendering}

Until now, the reseachers have been focusing on differentiable light transport integral. 
These include solving geometric discontinuity problem on the integral\cite{li2018differentiable, loubet2019reparameterizing, zhang2020path}.

From all these works, the derivatives with respect to scene parameters are calculated by automatic differentiation.

Reverse-mode differentiation it is possible to calculate partial derivatives with respect to arbitrary scene parameters. MoreThe multiple scene parameters can be optimized at once by reverse-mode differentiation.

For some parameters or joint optimization, more techniques.

\subsection*{Laplacian smooth gradient descent}

Stochastic gradient descent (SGD), 

machine learning paper 
smoothing method for stochastic gradient descent method (SGD) or GD
multiplying the usual (stochastic) gradient by a one-dimensional discrete Laplacian

\subsection*{Laplacian smoothing in computer graphics}

In geometry processing, the laplacian operator has been the key tool for solving geometry problem. Though possible many properties, discrete laplacians are enough.

Nicolet et al\cite{Nicolet2021Large} show gradient method. This works, mesh as scene parameter result
~ apply, then improve the convergence speed and high quality result.

This form can be interpreted as various meaning. As mentioned before, 
also, in differentiable rendering, in the 
we can think as laplacian smooth gradient using mesh propertices. 
Quasi-Newton's method for gradient descent using the mesh properties

From the point of view of geometry processing,
Or, on each iteration, filtering noisy vertices from the target mesh before gradient descent step.







