\section*{Project Description}\label{ch:ch1label}

We would like to focus on a popular macroeconomics problem called Utility Maximization, where the goal is to maximize the happiness of a consumer. The utility function will be denoted by $ U(c,l) $, which relates the consumption and the leisure of an individual in society with his/her happiness. A simple utility function is log-sum of consumption and leisure. Often, a discount factor $\beta$ (a modeling constant) is applied exponentially. Discount factor is used to model gradually decreased effects of future wealth on today \cite{Macroeconomics}.

The function takes two arguments: $ c \in \mathbb{R}^n = (c_1,c_2,...,c_n)$ and $ l \in \mathbb{R}^n = (l_1,l_2,...,l_n) $ which denote the normalized $(c_t,l_t \in [0,1])$ consumption and labor. The price for the single consumption good is denoted as $ P = (P_1,P_2,...,P_n)  $. The consumer, at time step t, needs to prepare $m_t = P_tc_t$ amount of money for the next time step, $t+1$, which is called the cash-in-advance constraint. There is the bonds mechanism, on time step t, when a consumer buys bonds worth $s_t$, at next time step he/she receives $(1+R)s_t$ with R being the interest rate. The same applies while borrowing. The consumption and work hours spent (labor) is related with a production function, which is in general a concave function such as $ln(l_t)$ or $l_t^\alpha$ with $0 \leq \alpha \leq 1$ being a constant. To keep things simple, we have assumed R is not fluctuating.

To simplify the problem, there is a bold assumption of "representative household" and we can employ that. This basically means that rather than modeling and optimizing for a single household, it is possible to optimize for a "representative" which is an arithmetic mean of the entire population. This approach introduces two more constraints, known as market-clearing constraints: $ s_t = 0 $ since on average, no-one can borrow without a lender and 
$ c_t = f(l_t)  $ since on average, a good which is not produced cannot be consumed. \\
\\The formal (mathematical) statement of our problem is:
\begin{equation*}
\begin{aligned}
& \underset{ \{ c_t,l_t,s_{t+1},m_{t+1} \} }{\text{maximize}}
& & U(c,l) = \sum_{t=0}^{N-}{\beta^t}[ln(c_t)+ln(1-l_t)] \\
& \text{subject to}
& & P_tc_t=m_t\\
& & & m_{t+1}+s_{t+1}=m_t+(1+R_t)s_t+P_tl_t+\tau_t-P_tc_t \\
& & & c_t = l_t \quad \text{to see effects of production function to optimization problem} \\
& & & s_t = 0
\end{aligned}
\end{equation*}
The formal version of problem is generally not solved numerically, instead it is solved algebraically and then expressions for terms are found and comments are made on them. However, in this project, after those steps, we would like to carry the problem to the numeric "domain" and visually see how environment parameters affect consumer decisions. In order to be able to solve with MATLAB, we need to add 2 more constraints, to bound c and l (normalize them). These constraints are:
\begin{align*}
0 \preceq c \preceq 1 \textrm{ and } 0 \preceq l \preceq 1
\end{align*}
These will not be derived in duality \& kkt sections however their effects will be expressed within simulation section.
