% !TEX root=./main.tex

\documentclass[aspectratio=169]{beamer}

\usecolortheme{seahorse}

\usepackage{standalone}
\usepackage{amsmath} % for matrix example
\usepackage{upquote} % so single quotes can be directly pasted
\usepackage{xcolor}
\usepackage{pgfpages} % for seperating pages
\usepackage{algorithm,algorithmic}
\usepackage{minted}
\usepackage{multicol}
\usepackage{subfigure}
\usepackage{graphicx}
\usepackage[thinc]{esdiff}

\setbeameroption{show notes}
\setbeameroption{show notes on second screen=right}

\title{"Applying gradient filtering on joint optimization"}
\subtitle{Photo Realistic Rendering}
\author{Hyeonjang An}

\institute{
    GIST
}

\begin{document}
\setbeamertemplate{caption}{\raggedright\insertcaption\par}
\begin{frame}
\titlepage

\end{frame}

\begin{frame}{Outline}
    \tableofcontents
\end{frame}

\section{Intro}
    \subsection{Problem statement and Idea}
    \subsection{Related Work}
\section{Method}
    \subsection{Implementation of the previous work: Mesh case}
    \subsection{Laplacian operator on the image kernel}
    \subsection{The choice on step length}
\section{Result}
\section{Conclusion}

% 1. Problem statement
% !TEX root=../main.tex
\documentclass[beamer]{standalone}
\begin{document}

% introduction
\begin{frame}{Problem statement and Idea}
    \begin{itemize}
        \setlength\itemsep{1em}

        \item Problem 
        \begin{enumerate}
            \item ill-posed problem
            \item noisy gradient descent
            \begin{itemize}
                \item Monte-carlo differentiable rendering
                \item Stochastic gradient descent
            \end{itemize}
        \end{enumerate}
        \item Motivation
        \begin{enumerate}
            \item Observing gradient filtering behaviour on the joint optimization problem \\ 
            (texture and envmap)
        \end{enumerate}
        \begin{figure}
            \includegraphics{./figures/Intro-1.png}
        \end{figure}
    \end{itemize}
\end{frame}
\end{document}

% 2. Related work explanation
% !TEX root=../main.tex
\documentclass[beamer]{standalone}
\begin{document}

% Differentiable rendering
\begin{frame}{Related-work}
\framesubtitle{Gradient Filtering}

\begin{itemize}
    \item "Laplacian Smooth Gradient Descent", Osher et al, 2019 ICLR
    \begin{figure}[h]
        \includegraphics[width=0.8\linewidth]{./figures/related-work-1.png}
    \end{figure}
    \begin{enumerate}
        \item machine learning paper 
        \item smoothing method for stochastic gradient descent method (SGD) or GD
        \item multiplying the usual (stochastic) gradient by a one-dimensional discrete Laplacian
    \end{enumerate}

\end{itemize}

\note[item] {

    }
\end{frame}

% Mesh propertices / laplace
\begin{frame}{Related-work}
\framesubtitle{Gradient Filtering}
\begin{itemize}
    \item "Large steps in the Inverse Rendering", \\ 
    Baptiste Nicolet, Alec Jacobson and Wenzel Jackob, 2021 SIGGRAPH ASIA
    \begin{figure}[h]
        \includegraphics[width=0.8\linewidth]{./figures/related-work-2.png}
    \end{figure}
    \begin{enumerate}
        \item Quasi-Newton's method for gradient descent using the mesh properties
        \item Main idea: gradient filtering with the properties of paramters
    \end{enumerate}
\end{itemize}
    
% notes %
\note[item]{
    }   
\end{frame}
\end{document}


% 3-1.method part
% !TEX root=../main.tex
\documentclass[beamer]{standalone}
\begin{document}
% ==============================================================
% Method intro
% ==============================================================
\begin{frame}{Method}
    \begin{itemize}
    \item Main goal is to determine a methodology 
    for taking large descent steps in mesh optimizing problems 
    using differentiable rendering
    \begin{figure}
        \begin{equation}
            minimize_{x\in\mathbb{R}^{nX3}}  \; \Phi(R(x))
        \end{equation}
    \caption{where $x$ is collection of vertex positions, $R$ is rendering function, $\Phi$ is a loss function}
    \end{figure}
\end{itemize}

% note %
\note[item]{
    As before said, the main goal of this paper is showing how to determine a methodology for taking large steps.
    So, the objective function could be like here.
    x is the vertex positions, R is rendering function and phi is loss function
}
\end{frame}



\end{document}

% 3-2 method part
% !TEX root=../main.tex
\documentclass[beamer]{standalone}
\begin{document}
% ==============================================================
% Method : Gradient Descenet Explanations - 0 
% ==============================================================   
\begin{frame}{Method}
    \framesubtitle{Laplacian operator on the image kernel}
        \begin{figure}[t]
        \centering
            \includegraphics[width=6cm]{./figures/image-kernel-laplacian.png}
            \caption{from: \emph{"A Tour of Modern Image Filtering"}}
        \end{figure}
        \begin{itemize}
            \item Filtering problem, the convenient vector form $z(x)=Wy$
        \end{itemize}
        \begin{table}
            \centering
            \resizebox{12cm}{!}{
            \begin{tabular}{ | c | c | c | c | } \hline
                Graph Laplacian & Symmetric                & DC eigenvector & Spectral Range \\ \hline
                Un-normalized   & $D - K$                  & Yes            & [0, n] \\ \hline
                Normalized      & $I - D^{-1/2}KD^{-1/2}$  & No             & [0, 2] \\ \hline  
            \end{tabular}}
        \end{table}

    % note %
    \note[item] {
    }
\end{frame}

\begin{frame}{Method}
    \framesubtitle{The choice on step length}
        \begin{equation}
            x \leftarrow x - \eta \textcolor{red}{(I + \lambda L)^{-p}} {\diffp{\Phi}{x}}
        \end{equation}
        \begin{itemize}    
            \item Need for choosing step size $\lambda$
            \item Step size can be chosen with weak wolfe condiiton, goldstein condition and backtracking ..
            \begin{itemize}
                \item But, we need the function evaluation for each step length
                \item In the current case, our function is MSE between target and rendering image
            \end{itemize}
            \item Now just using the scaled step size from an original adam optimizer
            \begin{itemize}
                \item On the "Large steps in the inverse rendering of Geometry" $\lambda$ is set as from 15 to 50
                \item On the "Laplacian Smooth Gradient Descent", the authors shows a proof on optimal value
            \end{itemize}
        \end{itemize}
\end{frame}

\end{document}

% 3-3 method part
% !TEX root=../main.tex
\documentclass[beamer]{standalone}
\begin{document}
% ==============================================================
% Method : Gradient Descenet Explanations - 7 (Original Adam vs Uniform Adam)
% ============================================================== 
\begin{frame}{Method}
\framesubtitle{Momentum and Variance}

\begin{columns}[T] % align columns
\begin{column}{.45\textwidth}
\begin{algorithm}[H]
    \begin{algorithmic}[1]
        \WHILE{$x$ not converged do}
            \STATE $g \leftarrow (I + \lambda L)^{-1} \diffp{\Phi}{x}$
            \STATE $m_{1} \leftarrow \beta_{1} m_{1} + (1-\beta_{1})g$
            \STATE $m_{2} \leftarrow \beta_{1} m_{2} + (1-\beta_{1})g^2$
            \STATE $u \leftarrow u-\frac{\eta}{(1-\beta_{1}^{k}) \sqrt{\frac{\left\lVert m_{2} \right\rVert_{\infty}}{1-\beta_{2}^{k}}}}$
        \ENDWHILE
        \STATE where $x(u)=(I+\lambda L)^{-1}u$
        \end{algorithmic}
    \caption{UniformAdam}
    \label{alg:seq}
    \end{algorithm}
\end{column}
\begin{column}{0.45\textwidth}
    \begin{figure}
        \includegraphics[width=\textwidth]{figures/method3-figure-1.png}
    \end{figure}
\end{column}
\end{columns}

% note %
\note[item] {
    Until now, I have explained the main idea of this paper.

    This is final form their implementation of Adam.

    The authors said that they discovered that aggresive steps disturb the smoothness of result.
    So, they modify it to a more uniform adaptation, and the picture of rightside is the comparison.
}
\end{frame}

% ==============================================================
% Method : Gradient Descenet Explanations - other details (implementation, )
% ============================================================== 
\begin{frame}{Method}
\framesubtitle{Remeshing}
\begin{itemize}
    \item Remeshing
    \begin{enumerate}
        \item remeshing is the technique that rebalance the energy of mesh
        \item In this paper, the authors adapted isotropic remeshing (Bostch and Kobbelt [2004])
        \item remeshing at manually specified timesteps
    \end{enumerate}
    \begin{figure}
        \includegraphics[width=0.5\textwidth]{figures/method3-figure-2.png}
    \end{figure}
\end{itemize}

% note %
\note[item] {
    And lastly, they use remeshing technique to solve the mesh energy balancing and resolution problem.
    
    Remeshing is the technique that rebalance the energy of mesh, by resampling vertices and modifying edge length.
    There are several remeshing technique, in this paper they adapt isotropic remeshing from Bostch and Kobbelt, the original paper title is "A Remeshing Approach to Multiresolution Modeling"
} 

\end{frame}
\end{document}

% 4. conclusion
% !TEX root=../main.tex
\documentclass[beamer]{standalone}
\begin{document}
\begin{frame}{Summary}

\begin{itemize}
    \setlength\itemsep{1em}
    \item Gradient filtering using the known propertices on target parameters
    \item Test on the joint optimization problem, texture case
    \item Due to the bias (with the rougly designed gradient filtering), \\ 
    optimizing enviroment map shows a awful result
\end{itemize}
\end{frame}

\begin{frame}{Conclusion}

    \begin{itemize}
        \setlength\itemsep{1em}
        \item Gradient filtering makes the convergence faster
        \begin{itemize}
            \item reduce noise
            \item give a bigger step length
        \end{itemize}
        \item to global minimum, when properly applied 
        \item Not likely mesh, bias becomes bigger problem in the texture optimization procedure
    \end{itemize}
    
        % notes %
        \note[item]{
        }
        
    \end{frame}

    \begin{frame}{Q\&A}

        \note[item] {
            this is the end of the presentation
        }
    \end{frame}
\end{document}

\end{document}